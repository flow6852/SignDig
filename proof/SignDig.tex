\documentclass[14pt]{jsarticle}
\usepackage[top=15truemm,bottom=15truemm,left=15truemm,right=15truemm]{geometry}
\usepackage[all]{xy}
\usepackage{amsthm,mathtools,amssymb,graphicx,here, amsmath}
\pagestyle{plain}

\theoremstyle{definition}
\newtheorem{dfn}{Def.}[subsection]
\newtheorem{thm}[dfn]{Thm.}
\newtheorem{lem}[dfn]{Lem.}
\newtheorem{cor}[dfn]{Cor.}
\newtheorem{noti}[dfn]{Notificate.}
\newtheorem{doril}[dfn]{演習問題}

\begin{document}
\begin{proof}
$a ,b\in {\mathbb R}$ とする.
ある数$r\in {\mathbb R}$について,$n$桁目まで等しい有限小数を$\overline r$とする.
また,$r = {\overline r} + \varepsilon_r$を満たす$\varepsilon_r$を取る.すると以下の条件を満たす. \\
$$
\frac{1}{2}r = \begin{cases}
	\frac{1}{2}\overline r + \frac{1}{2}\varepsilon_r (rのn桁目が偶数) \\
	\frac{1}{2}\overline r + 5\times10^{-(n+1)} + \frac{1}{2}\varepsilon_r (rのn桁目が奇数)
	\end{cases}
$$
ここで,$$
\frac{1}{2}{\overline r} = \begin{cases}
	{\overline {\frac{1}{2}r}} ({\overline r}のn桁目が偶数) \\
	{\overline {\frac{1}{2}r}} + 5\times 10^{-(n+1)} ({\overline r}のn桁目が奇数)
\end{cases}
$$
が言える.また,$a, b\in{\mathbb R}$について

\textcircled{\scriptsize 1}$a,b$の$n$桁目が共に偶数のとき\\
$\overline a, \overline b$の$n$桁目も偶数であるから,
\begin{eqnarray}
\left| \frac{1}{2}a + \frac{1}{2}b - \overline{(\frac{1}{2}\overline a + \frac{1}{2}\overline b)} \right| &=&
\left| \frac{\overline a}{2} + \frac{\varepsilon_a}{2} + \frac{\overline b}{2} + \frac{\varepsilon_b}{2} 
     - \overline{\frac{\overline a}{2} + \frac{\overline b}{2}} \right| \\ &=& 
\left| \overline{\frac{\overline a}{2} + \frac{\overline b}{2}}
     + \frac{\varepsilon_a}{2} + \frac{\varepsilon_b}{2}
     - \overline{\frac{\overline a}{2} + \frac{\overline b}{2}} \right| \\ &=&
\left| \frac{\varepsilon_a}{2} +\frac{\varepsilon_b}{2} \right| 
\end{eqnarray}
\textcircled{\scriptsize あ}$\varepsilon_a < \varepsilon_b$のとき
\begin{eqnarray}
a + \frac{1}{2}(b - a) &=& \overline a + \varepsilon_a + \frac{1}{2}(\overline b + \varepsilon_b - \overline a - \varepsilon_a) \\
                       &=& \overline a + \varepsilon_a
                         + \frac{1}{2}(\overline b - \overline a) + \frac{1}{2}\varepsilon_b - \frac{1}{2}\varepsilon_a \\
		       &=& \overline a + \varepsilon_a
                         + \frac{1}{2}\overline{(\overline b - \overline a)} 
                         + \frac{1}{2}\varepsilon_b - \frac{1}{2}\varepsilon_a \\
                       &=& \overline a 
                         + \overline{\frac{1}{2}(\overline b - \overline a)} 
                         + \frac{1}{2}\varepsilon_b + \frac{1}{2}\varepsilon_a \\
		       &=& \overline{\overline a 
                         + \overline{\frac{1}{2}(\overline b - \overline a)}}
                         + \frac{1}{2}\varepsilon_b + \frac{1}{2}\varepsilon_a
\end{eqnarray}
だから
\begin{eqnarray}
\left| a + \frac{1}{2}(b - a) - (\overline{\overline a + \frac{1}{2}(\overline b - \overline a})) \right| &=&
\left| \frac{1}{2}\varepsilon_b + \frac{1}{2}\varepsilon_a \right|
\end{eqnarray}
が成り立つから,(3),(9)より,数値誤差は等しい.\\
\textcircled{\scriptsize い}$\varepsilon_a > \varepsilon_b$のとき
\begin{eqnarray}
a + \frac{1}{2}(b - a) &=& \overline a + \varepsilon_a 
                         + \frac{1}{2}(\overline{(b-1.0\times10^{-n})} + 1.0\times10^{-n} + \varepsilon_b - \overline a - \varepsilon_a) \\
                       &=& \overline a + \varepsilon_a
                         + \frac{1}{2}(\overline b - \overline a) + \frac{1}{2}\varepsilon_b - \frac{1}{2}\varepsilon_a \\
		       &=& \overline a + \varepsilon_a
                         + \frac{1}{2}\overline{(\overline b - \overline a)} 
                         + \frac{1}{2}\varepsilon_b - \frac{1}{2}\varepsilon_a \\
                       &=& \overline a 
                         + \overline{\frac{1}{2}(\overline b - \overline a)} 
                         + \frac{1}{2}\varepsilon_b + \frac{1}{2}\varepsilon_a \\
		       &=& \overline{\overline a 
                         + \overline{\frac{1}{2}(\overline b - \overline a)}}
                         + \frac{1}{2}\varepsilon_b + \frac{1}{2}\varepsilon_a
\end{eqnarray}
\textcircled{\scriptsize 2}
\textcircled{\scriptsize 3}
\textcircled{\scriptsize 4}
\end{proof}
\end{document}
